\section{Introduction }	
\subsection{Overview}
\begin{itemize}
	\item Supervised learning = learning with labels defined by human; Unsupervised learning = finding patterns in data. Reinforcement learning is a 3rd machine learning paradigm, in which the agent tries to maximise its reward signal.
	\item Exploration versus exploitation problem - agent wants to do what it has already done to maximise reward by exploitation, but there may be a bigger reward available if it were to explore.
	\item RL is based on the model of human learning, similar to that of the brain's reward system.
\end{itemize}

\subsection{Elements of Reinforcement Learning}
\begin{description}
	\item[Policy] Defines the agent's way of behaving at any given time. It is a mapping from the perceived states of the environment to actions to be taken when in those states.
	\item[Reward Signal] The reward defines the goal of the reinforcement learning problem. At each time step, the environment sends the RL agent a single number, a \textit{reward}. It is the agent's sole objective to maximise this reward. In a biological system, we might think of rewards as analogous to pain and pleasure. The reward sent at any time depends on the agent's current action and the agent's current state. If my state is hungry and I choose the action of eating, I receive positive reward.
	\item[Value function] Reward functions indicate what is good in the immediately, but value functions specify what is good in the long run. The value function is the total expected reward an agent is likely to accumulate in the future, starting from a given state. E.g. a state might always yield a low immediate reward, but is normally followed by a string of states that yield high reward. Or the reverse. Rewards are, in a sense, primary, whereas values, as predictions of rewards, are secondary. Without rewards there could be no value. Nevertheless it is values with which we are most concerned when evaluating decisions. We seek actions that bring the highest value, not the highest reward, because they obtain the greatest amount of reward over the long run. Estimating values is not trivial, and efficiently and accurately estimating them is the core of RL.
	\item[Model of environment (optionally)] Something that mimics the behaviour of the true environment, to allow inferences to be made about how the environment will behave. Given a state and action, the model might predict the resultant next state and next reward. They are used for \textit{planning}, that is, deciding on a course of action by considering possible future situations before they are actually experienced. 
\end{description}

	
	
	
	
